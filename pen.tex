Жарким летним вечерком,
Средь дубрав пахучих,
Стаи смелых мотыльков
Разгоняли тучи.

«Уходите, уходите!»
Вслед они пищали
Дождь с собою заберите,
Ну а с ним печали.

Под деревьями в тени,
Попивая пиво,
Стихоплет перо чинил
И давался диву.

Мотыльков не слышал он,
Занят был работой,
Прогоняя сладкий сон,
И бранясь с дремотой.

Только всё не шло на лад
Скучное занятье,
Он и бросить его рад,
Пробурчав проклятье.

Это чудное перо
На заре купил он,
Не залив с утра в нутро,
Пары кружек пива.

Щедро продавец сулил
Муз и вдохновенье,
«С тем пером» - он говорил:
«Обретешь терпенье,

Вис напишешь ты несчесть,
Славой длань покроешь.
Обратишь чужую лесть
Золотой горою».

Вот и стал наш стихоплёт
Над пером возиться,
Так возьмет, и так возьмет!
Перо-то не годится!

Разозлился он тогда,
Гулко оземь топнул.
«Не выходит ни черта!
Чтоб я тут же лопнул!»

И заплакал от тоски
Средь дубрав пахучих,
Где ночные мотыльки,
Разгоняют тучи.



Frakhtan-teh: А где в ДС можно добыть всякие Linux-сувениры, кроме как на заказ?
Compas: Грабли можно в хозтоварах купить.
Dragon59: И велосипеды в спортивном.
